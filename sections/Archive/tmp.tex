Figure~XX can be discussed most clearly by reducing the circuit to its dominant small‑signal elements. Each tri‑stated input inverter contributes the usual gate capacitance \(C_{\text{in}}\) to ground. When an external transition drives the pad, that capacitance sources (or sinks) a displacement current \(i = C_{\text{in}}\frac{\mathrm{d}v}{\mathrm{d}t}\) into the first internal node \(N_{1}\). Displacement current of this sort is inevitable even in the high‑impedance state.

Node \(N_{1}\) sees two parallel paths: the effective on‑resistance \(R_{\text{ON}}\) of whichever pass device (pull‑up or pull‑down) is enabled, and the input capacitance of the next stage, \(C_{\text{next}}\). For frequencies below the corner
\[
f_{\text{iso}} \approx \frac{1}{2\pi R_{\text{ON}} C_{\text{next}}},
\]
the capacitive reactance \(\lvert Z_{C_{\text{next}}}\rvert\) is far larger than \(R_{\text{ON}}\), so almost the entire displacement current is shunted through the resistive path to the supply rails. Only a static droop \(\Delta V = i R_{\text{ON}}\) appears at \(N_{1}\). As the excitation 

The crucial point is that both the second stage and the output drivers are fully differential. Any charge that leaks through \(C_{\text{next}}\) appears as \emph{common‑mode} on the complementary pair at node \(N_{2}\), while the useful signal path only amplifies the \emph{differential} component. With typical on‑chip matching, the common‑mode‑rejection ratio (CMRR) of such a pair exceeds 30–40 dB, so the residual feed‑through at the final output is attenuated by the same margin. Even the small \(\Delta V\) produced across \(R_{\text{ON}}\) modulates the following transistors via their transconductance \(g_{m}\), but this too is injected symmetrically and is largely rejected for the same reason. In practice, then, the overall input‑to‑output isolation is set by \(\beta/\text{CMRR}\), which is well below one per cent until the excitation spectrum approaches the multi‑GHz regime where \(C_{\text{next}}\) can no longer be ignored.
