This thesis investigates the design of an on‑chip eight‑phase clock generator for high‑speed Serializer-Deserializer (\gls{serdes}) applications.  Driven by the explosive growth in data‑centre traffic and bandwidth‑hungry services, modern \gls{serdes} must operate above 20 GHz with sub‑picosecond timing accuracy.  Conventional multi‑phase generation methods, delay-locked-loop (\gls{dll}) chains, injection‑locked ring oscillators and phase-interpolator (\gls{pi}) networks, provide useful trade‑offs in resolution, jitter and power but struggle to scale beyond 10 GHz or require complex calibration.  To meet next‑generation requirements, a \gls{pde} was developed in TSMC 3 nm FinFET technology.  The programmable-delay element (\gls{pde}) combines coarse switched‑capacitor tuning with fine bias adjustment to generate eight evenly spaced phases from 5 GHz to 22.5 GHz.  A behavioural Verilog‑A calibration loop accelerates simulation and adapts the delay element across process, voltage and temperature (\gls{pvt}) corners.

The thesis begins by reviewing state‑of‑the‑art multi‑phase generators and programmable delay circuits, highlighting the limitations of current‑starved inverters and shunt‑capacitor delay lines.  It then details the design methodology, including transistor characterisation, architectural exploration and simulation framework.  Several architectures were explored, switched‑capacitor delay elements, feed‑forward phase interpolation and supply‑bias tuning, before converging on a pure delay‑line solution.  The final architecture uses eight independent delay paths for high‑frequency operation and incorporates mid‑ and low‑frequency modes via clock division.  Input demultiplexers and output multiplexers select the appropriate path, while fine current‑starved inverters provide residual skew correction.

Simulation results demonstrate that the proposed multi‑phase generator achieves less than 35 fs end‑to‑end jitter and approximately 1° phase accuracy at 22.5 GHz, meeting the target specifications.  Power consumption is around 60 mW in high‑frequency mode and decreases to roughly 15 mW in low‑frequency mode.  The design shows robust operation across \gls{pvt} variations and offers a versatile solution for next‑generation \gls{serdes} clocking.  Future work may extend the fine‑tuning resolution through additional control bits and integrate the design into a complete transmitter chain.
