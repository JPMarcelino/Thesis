\section{Derivation of the \texorpdfstring{$\ln 2$}{ln 2} factor}
\label{app:ln2_derivation}

Assume a linear, time\textendash invariant RC network so that the step response of the
node voltage is exponential. For the rising edge (charging through
$R_{\uparrow}$) we have:

\[
V_{\text{rise}}(t) = V_{\max}\!\left(1-e^{-t/\tau_{\uparrow}}\right),
\qquad
\tau_{\uparrow}=R_{\uparrow}C_{\mathrm{eff}}
\]

The sampler triggers when the waveform reaches one\textendash half of its final value,
i.e. when $V_{\text{rise}}(t_{\uparrow})=\tfrac{1}{2}V_{\max}$. Setting
the two expressions equal gives:

\[
\frac{1}{2}V_{\max}
      = V_{\max}\left(1-e^{-t_{\uparrow}/\tau_{\uparrow}}\right)
\Longrightarrow
e^{-t_{\uparrow}/\tau_{\uparrow}} = \frac{1}{2}
\]

Taking the natural logarithm of both sides:

\[
-\frac{t_{\uparrow}}{\tau_{\uparrow}}
      = \ln\!\left(\tfrac{1}{2}\right)
\Longrightarrow
t_{\uparrow}= \tau_{\uparrow}\ln 2
            = R_{\uparrow}C_{\mathrm{eff}}\ln 2
\tag{A}
\]

For the falling edge (discharging through $R_{\downarrow}$) the waveform is:

\[
V_{\text{fall}}(t) = V_{\max}e^{-t/\tau_{\downarrow}},
\qquad
\tau_{\downarrow}=R_{\downarrow}C_{\mathrm{eff}}
\]

The $50\,\%$ crossing occurs when $V_{\text{fall}}(t_{\downarrow})=
\tfrac{1}{2}V_{\max}$:

\[
\frac{1}{2}V_{\max}
      = V_{\max}e^{-t_{\downarrow}/\tau_{\downarrow}}
\Longrightarrow
t_{\downarrow}= \tau_{\downarrow}\ln 2
              = R_{\downarrow}C_{\mathrm{eff}}\ln 2
\tag{B}
\]

Equations (A) and (B) are the origin of the $\ln 2$ factor that
appears in the timing expressions and in the incremental delay formulae.

