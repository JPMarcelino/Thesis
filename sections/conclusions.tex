\section{Summary and Key Achievements}
This thesis investigated the design of a programmable delay element and multi-phase clock generator capable of sustaining 45\degree{} phase spacing up to \SI{22.5}{\giga\hertz} in TSMC's \SI{3}{\nano\metre} FinFET process. Building on the earlier current-starved delay stage, the work introduced a hierarchy of digitally assisted calibration loops that decouple coarse capacitance tuning from fine bias trimming. The Verilog-A infrastructure developed in Chapter~\ref{sec:RTL_tuning} enabled exhaustive verification across \gls{pvt} space, allowing rapid exploration of architectural trade-offs and guiding the transition from the legacy 7~nm design to the 3~nm implementation.

Successive architectural refinements were assessed in schematic and behavioural simulation. Switched-capacitor and phase-interpolator approaches quantified the limitations imposed by slew-rate modulation, area overhead and jitter sensitivity. The feed-forward architecture demonstrated how internal phase reinforcement can extend tuning range, but detailed \gls{pvt} analysis exposed strong inter-branch coupling that jeopardised stability of the \ang{135} and \ang{315} paths. The final pure-delay topology eliminated that dependency by restoring eight fully independent delay lines while preserving the coarse/fine calibration hierarchy and multi-modal clocking network.

The resulting \gls{mpg} delivers eight clock phases over the \SI{5}{\giga\hertz}--\SI{22.5}{\giga\hertz} span with code margins of roughly five steps at 22.5~GHz, limiting the risk of calibration saturation across \gls{pvt}. End-to-end simulations that include the input demultiplexer, multiphase generator, output multiplexer and parked fine-tuning stages report an integrated jitter below \SI{35}{\femto\second}. The three operating modes meet their power targets with approximately \SI{60}{\milli\watt}, \SI{40}{\milli\watt} and \SI{15}{\milli\watt} consumption in HF, MF and LF modes, respectively. Together, these results validate the methodology and establish a solid platform for high-speed \gls{serdes} clock distribution in advanced CMOS nodes.

\section{Future Perspectives}
Although the present design largely satisfies the functional requirements, several avenues remain open for further improvement. First, the fine-tuning infrastructure should be revisited to close the residual gap to the resolution target. Extending the phase-interpolator control word, or hybridising it with current-starved bias trim, promises finer resolution without reintroducing the duty-cycle perturbations observed when toggling large capacitor codes or tuning supply voltages. Second, the available five-code margin must be validated under Monte Carlo and ageing analyses.

A third priority is to complete a layout-driven verification flow. The exploratory back-annotation study confirmed that the wrapper models capture a significant portion of the parasitics, yet a full custom layout, parasitic extraction and sign-off simulation are required to quantify metal-level coupling and routing-induced skew. 

Finally, migrating the Verilog-A calibration loops to synthesizable RTL and integrating them with the system controller will enable hardware validation, including built-in self-test hooks for on-silicon phase observability. 

Addressing these points will pave the way to silicon implementation and ensure the robustness demanded by next-generation \gls{serdes} interfaces.
