The increasing demand for higher data rates in modern integrated circuits (ICs), particularly input-output (\gls{io}) blocks such as serializer-deserializer (\gls{serdes}) applications, has driven the development of innovative circuit designs capable of operating under stringent timing constraints~\cite{horowitz2005scaling,seok2010progress}. The volume of data created and consumed is growing at a phenomenal rate, with over 180 zettabytes of data expected to be generated in 2025~\cite{consultancy2024data}. In part, these developments are driven by over-the-top (\gls{ott}) services such as Netflix and YouTube, which have become increasingly popular in recent years and demand high-bandwidth, low-latency connections. Artificial intelligence (\gls{ai}) and machine-learning (\gls{ml}) applications also require high-speed data transfer to process large datasets efficiently. As a result, the demand for \gls{serdes} circuits capable of operating at high frequencies has surged.

\gls{serdes} circuits are important components of modern ICs, enabling efficient data transfer across long distances and between different clock domains. These circuits convert parallel data streams into high-speed serial signals and subsequently reconstruct the original data at the receiving end. Datacentres and high-performance computing applications rely on \gls{serdes} circuits to achieve high data rates and low power dissipation. \gls{serdes} circuits commonly operate at frequencies exceeding 20 GHz, where timing mismatches as small as femtoseconds can degrade performance. Precise alignment of clock and data signals is essential to avoid bit errors caused by jitter or crosstalk~\cite{lee2011self,nakamura2022high}. Advanced \gls{serdes} architectures require the use of equalization, complex clock recovery and dynamic, programmable timing calibrations to compensate for loss and distortion.

Time-interleaving (TI) and phase-shifting techniques are commonly used to achieve higher effective clock frequencies and improve data throughput~\cite{Razavi2009PLL}. Time interleaving splits a single ultra high-speed data stream into multiple parallel lower speed paths, which process the data concurrently, overcoming traditional bandwidth limitations. However, achieving precise phase shifts and maintaining signal integrity at these frequencies poses significant challenges, particularly in advanced, deep-submicron CMOS nodes like 7~nm and 3~nm~\cite{loke2019nanoscale,caignet2001challenge}.

The increasing data rates will predictably lead to a saturation point without constant paradigm shifts and innovation in circuit design. Strategies such as PAM-8 modulation schemes are being explored to increase data throughput without increasing the clock frequency. These modulation schemes are designed to increase the number of bits transmitted per symbol reducing the number of clock cycles required to transmit the same amount of data. Naturally, these schemes require precisely tuned clock distribution as tolerances reduce.

This thesis thus aims to explore the design of a programmable delay element (\gls{pde}) that can be used in high-speed \gls{serdes} applications. The design focuses on achieving high resolution and low jitter performance across a range of frequencies from 5~GHz to 22.5~GHz.
