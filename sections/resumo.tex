Esta dissertação analisa um gerador de relógio multifásico integrado com oito fases para ligações Serializer-Deserializer (SerDes). O crescimento do tráfego nos centros de dados obriga os SerDes modernos a operar acima de 20~GHz com precisão sub-picosegundo. Soluções clássicas como cadeias de atraso bloqueadas, osciladores de anel bloqueados por injeção e redes de interpolação de fase oferecem compromissos úteis, mas não escalam além de 10~GHz sem calibração. Propõe-se um elemento de atraso programável (programmable delay element, PDE) em tecnologia TSMC FinFET de 3~nm que combina bancos de condensadores de ajuste grosseiro com polarização fina para gerar oito fases entre 5 e 22{,}5~GHz. Um laço comportamental de calibração em Verilog-A acelera a verificação e adapta o PDE a variações de processo, tensão e temperatura (PVT).

A investigação abre com uma revisão do estado da arte em geradores multifásicos e circuitos de atraso, evidenciando limitações de inversores estrangulados por corrente e de linhas com condensadores shunt. A metodologia cobre caracterização de transistores, exploração arquitetural e modelação comportamental; as opções avaliadas incluem condensadores comutados, interpolação com realimentação direta e modulação de polarização, convergindo para uma linha de atraso pura com oito percursos independentes e modos médio/baixo obtidos por divisão de relógio.

As simulações indicam jitter inferior a 35~fs e precisão de fase próxima de $1^{\circ}$ a 22{,}5~GHz, com dissipações de cerca de 60~mW e 15~mW nos modos de alta e baixa frequência. O projeto mostra-se robusto face a variações PVT e adequado a SerDes de próxima geração; trabalhos futuros ampliarão a resolução do ajuste fino e integrarão o bloco numa cadeia completa de transmissor.

\medskip
\noindent\textbf{Palavras-chave:} gerador multifásico, elemento de atraso programável, SerDes, jitter, tecnologia FinFET.
