Esta dissertação investiga o projeto de um gerador de relógio em chip com oito fases para aplicações Serializer–Deserializer (SerDes) de alta velocidade.  Impulsionados pelo crescimento explosivo do tráfego em centros de dados e de serviços com grande largura de banda, os SerDes modernos têm de operar acima de 20 GHz com precisão temporal subpicosegundo.  Os métodos convencionais de geração multifásica, ciclos bloqueados por atraso, osciladores de anel bloqueados por injecção e redes de interpolação de fase, oferecem compromissos úteis em resolução, jitter e consumo, mas têm dificuldade em escalar além dos 10 GHz ou exigem calibração complexa.  Para satisfazer os requisitos da próxima geração, foi desenvolvido um elemento de atraso programável (PDE, do inglês Programmable Delay Element) em tecnologia TSMC de 3 nm FinFET.  O PDE combina afinação grosseira com bancos de condensadores com ajuste fino da polarização para gerar oito fases equiespaciadas de 5 GHz a 22,5 GHz.  Um laço de calibração comportamental em Verilog‑A acelera as simulações e adapta o elemento de atraso às variações de processo, tensão e temperatura PVT (do inglês Process, Voltage, Temperature).

A tese começa com uma revisão do estado da arte em geradores multifásicos e circuitos de atraso programável, destacando as limitações dos inversores estrangulados por corrente e das linhas de atraso baseadas em condensadores.  Em seguida descreve a metodologia de concepção, incluindo caracterização de transístores, exploração de arquitecturas e modelo comportamental.  Várias arquitecturas foram exploradas, elementos de atraso por condensadores comutados, interpolação de fase com realimentação directa e ajuste por polarização de alimentação, antes de se convergir para uma solução de linha de atraso pura.  A arquitectura final utiliza oito caminhos de atraso independentes para a operação de alta frequência e incorpora modos de média e baixa frequência através da divisão do relógio.  Desmultiplexadores de entrada e multiplexadores de saída seleccionam o caminho adequado, enquanto inversores de crrente limitada corrigem a defasagem residual.

Os resultados de simulação demonstram que o gerador de fases proposto atinge jitter inferior a 35 fs e precisão de fase de cerca de 1° a 22,5 GHz, cumprindo as especificações alvo.  A dissipação de potência é cerca de 60 mW em modo de alta frequência e diminui para cerca de 15 mW em modo de baixa frequência.  O Projeto demonstra funcionamento robusto em variações PVT e oferece uma solução versátil para o clocking de SerDes de próxima geração.  Trabalhos futuros podem permitir aumentar a resolução da afinação fina através de bits de controlo adicionais e integrar o desenho numa cadeia completa de transmissor.